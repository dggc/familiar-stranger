\section{Trabalhos relacionados}
\label{sec:related-work}

Aplicações sociais sensíveis à proximidade, que nesse trabalho
são generalizadas apenas para redes sociais móveis, é um assunto
relativamente novo na literatura. No entanto, atualmente é possível
encontrar vários trabalhos que apresentam aplicações que se baseiam
nesse assunto. Muitos desses trabalhos empregam o conceito de Estranho
Familiar. A seguir, serão apresentados alguns dos principais trabalhos
relacionados à aplicações sociais sensíveis à proximidade:

Em~\cite{SocialNet}, os autores
apresentam um sistema que foi denominado Social Net. O Social Net é um
sistema de casamento de interesses de usuários. O diferencial desse
projeto é que eles fazem a inferência de interesses dos usuários,
baseado nos padrões de localização dos usuários. Assim é possível
recomendar amizades a pessoas com interesses em comum de uma maneira
menos desagradável. Ao invés de alertar pessoas com perfis parecidos
assim que elas se encontram, independente do local, o trabalho propõe
realizar essa ação dentro de um contexto mais apropriado. A ideia é o
dispositivo informar para um usuário quando um amigo em comum está
próximo. Isso pode favorecer uma introdução.

O trabalho apresentado em~\cite{FamiliarStranger} propõe formas de
utilização do conceito de estranho familiar em um espaço urbano. Os
autores empregam no trabalho uma perspectiva urbanística de aplicação de
novas tecnologias para trabalhar com esse conceito, enxergando a cidade
não somente em termos espaciais, mas também temporais. O trabalho,
diferente de outros, não visa fazer um localizador de amigos, que tenta
converter estranhos em amigos. Os autores acreditam que ter estranhos em
nosso espaço urbano não é uma coisa negativa. Foram exemplificados dois
cenários que fogem um pouco dos cenários geralmente ressaltados por
outros artigos que também trabalham com o conceito de estranho familiar:
1- uma pessoa pode querer um local com maior privacidade, sem a presença
nem mesmo de estranhos familiares; 2- uma  pessoa em um local novo pode
se sentir mais confortável sabendo que estranhos familiares estão por
perto, mesmo não os reconhecendo, isso pode acontecer, por exemplo, em
uma nova vizinha para a  qual uma pessoa acabou de se mudar. Os autores
refizeram o experimento de Milgran, realizado em 1972, sobre estranhos
familiares. Com certeza esse fenômeno ainda existe, porém não é claro o
seu grau de intensidade. Apesar de atualmente existirem tecnologias para
uma avaliação mais precisa desse estudo, os autores usaram os mesmos
recursos do estudo original, entrevistas, avaliação fotográficas de
pessoas em ambientes públicos, etc. Nosso trabalho possui objetivos em
comum com esse. Pretende-se utilizar tecnologias atuais para melhor
entender o fenômeno de estranho familiar e identificar as melhores
formas de utilizá-lo.

O sistema apresentado em~\cite{SocialSerendipity}, denominado Serendipity,  apresenta um sistema que
identifica usuários que estão próximos um dos outros. Quando o
Serendipity descobre um novo dispositivo, ele automaticamente envia uma
mensagem para um ``servidor social''. Esse servidor é responsável por
encontrar um usuário com com perfil parecido. Quando isso é realizado, o
servidor se encarrega de apresentar esses usuários.

O trabalho apresentado em~\cite{Uttering}
faz um estudo da interação social através de micro-blogging sem a
utilização da Internet. O conteúdo é transmitido através de transmissões
diretas de um usuário para outro utilizando redes sem fio. Esse trabalho
apresenta vários desafios, um deles é a questão de redes DTN (Delay
Tolerant Networks).

Os autores de~\cite{SocialNetworking} desenvolveram uma extensão
do Facebook que foi denominada Cityware. A ideia dessa aplicação é
    mesclar informações disponíveis no Facebook, com traces de
    mobilidade capturados através de Bluetooth. Isso proporciona a
    diminuição entre da distância entre as redes sociais online e
    físicas. Os usuários possuem a possibilidade explorar uma combinação
    de suas redes sociais online com suas redes sociais físicas.

Em \cite{Hummingbird} os autores apresentam um dispositivo denominado
Hummingbird, que alerta quando existe um usuário perto de outro.
Hummingbird suporta colaboração e foi desenvolvido para melhorar formas
tradicionais de comunicação online, como mensagens instantâneas.

Foram apresentados nessa Seção apenas alguns dos principais trabalhos
encontrados na literatura, relacionados ao nosso tema. No decorrer do
trabalho serão revisados outros trabalhos com o intuito de
aprimorar essa Seção. 
