\section{Trabalhos relacionados}
\label{sec:related-work}

Nessa seção serão abordados duas classes de trabalhos. A primeira
considera alguns dos estudos que abordam codificação em
rede. Já a segunda aborda trabalhos que consideraram redes sociais
móveis.

Os artigos~\cite{NWC_theory} e~\cite{citeulike:1401816} foram um dos
primeiros a abordarem o tema codificação em rede. Como o assunto era
muito novo, o objetivo desses trabalhos era introduzir o tema. Foi
explicado o que é codificação em rede, como utilizá-la e quais os seus
benefícios.

Em~\cite{pedersen09:_networ_codin_mobil_devic_system} os autores
implementaram uma Random Linear Network Coding em dispositivos
(celulares) reais. Nesse trabalho é mostrado que essa técnica é viável
em dispositivos com limitações de capacidade computacionais e de
bateria. 

Em~\cite{citeulike:1401840} os autores realizam um interessante
trabalho, aplicando network coding em uma VANET (rede veicular). Além da
alta qualidade desse trabalho, um ponto interessante é a clareza com que
é abordado a técnica de codificação em rede. Esse trabalho foi de grande
relevância na implementação prática do trabalho aqui apresentado.

Para a realização deste trabalho foram lidos outros materiais sobre
network coding (uma vez que esse não é um tema trivial). No entanto,
optamos por abordar apenas alguns desses trabalhos.

Aplicações sociais sensíveis à proximidade, que nesse trabalho
são generalizadas apenas para redes sociais móveis, é um assunto
relativamente novo na literatura. No entanto, atualmente é possível
encontrar vários trabalhos que apresentam aplicações que se baseiam
nesse assunto. A seguir, serão apresentados alguns dos principais
trabalhos
relacionados à aplicações sociais sensíveis à proximidade:

Em~\cite{SocialNet}, os autores
apresentam um sistema que foi denominado Social Net. O Social Net é um
sistema de casamento de interesses de usuários. O diferencial desse
projeto é que eles fazem a inferência de interesses dos usuários,
baseado em seus padrões de localização. Assim, é possível
recomendar amizades a pessoas com interesses em comum de uma maneira
menos desagradável. Ao invés de alertar pessoas com perfis parecidos
assim que elas se encontram, independente do local, o trabalho propõe
realizar essa ação dentro de um contexto mais apropriado. A ideia é o
dispositivo informar para um usuário quando um amigo em comum está
próximo. Isso pode favorecer uma introdução.

O trabalho apresentado em~\cite{FamiliarStranger} propõe formas de
utilização do conceito de
estranho familiar em um espaço urbano. Um estranho
familiar~\cite{FamiliarStranger} é alguém que possui algum hábito
semelhante a outra pessoa, entretanto, que nunca se relacionou com a
mesma. O estudo deste conceito vem crescendo na área de redes sociais,
uma vez que estranhos
familiares geralmente possuem interesses e rotinas em comum, podendo
assim se beneficiar de um possível relacionamento. Os autores empregam
no trabalho
uma perspectiva urbanística de aplicação de novas tecnologias para
trabalhar com esse conceito, enxergando a cidade não somente em termos
espaciais, mas também temporais. O trabalho, diferente de outros, não
visa fazer um localizador de amigos, que tenta converter estranhos em
amigos. Os autores acreditam que ter estranhos em nosso espaço urbano
não é uma coisa negativa. Foram exemplificados dois cenários que fogem
um
pouco dos cenários geralmente ressaltados por outros artigos que também
trabalham com o conceito de estranho familiar:
\begin{itemize}
    \item Uma pessoa pode querer um local com maior privacidade, sem a presença nem mesmo de estranhos familiares;
    \item Uma  pessoa em um local novo pode se sentir mais confortável
    sabendo que estranhos familiares estão por  perto, mesmo não os
    reconhecendo; isso pode acontecer, por exemplo, em uma nova
    vizinhaça para a  qual uma pessoa acabou de se mudar.
\end{itemize}
Os autores refizeram o
experimento de Milgran, realizado em 1972, sobre estranhos familiares.
Com certeza esse fenômeno ainda existe, porém não é claro o seu grau de
intensidade. Apesar de atualmente existirem tecnologias para uma
avaliação mais precisa desse estudo, os autores usaram os mesmos
recursos do estudo original: entrevistas, avaliação fotográficas de
pessoas em ambientes públicos, etc. Nosso trabalho possui objetivos em
comum com esse. Pretende-se utilizar tecnologias atuais para melhor
entender
o fenômeno de estranho familiar e identificar as melhores formas de
utilizá-lo.

O sistema apresentado em~\cite{SocialSerendipity}, denominado
Serendipity,  apresenta um sistema que
identifica usuários que estão próximos um dos outros. Quando o
Serendipity descobre um novo dispositivo, ele automaticamente envia uma
mensagem para um ``servidor social''. Esse servidor é responsável por
encontrar um usuário com com perfil parecido. Quando isso é realizado, o
servidor se encarrega de apresentar esses usuários.

O trabalho apresentado em~\cite{Uttering}
faz um estudo da interação social através de micro-blogging sem a
utilização da Internet. O conteúdo é transmitido através de transmissões
diretas de um usuário para outro utilizando redes sem fio. Esse trabalho
apresenta vários desafios: um deles é a questão de redes DTN (Delay
Tolerant Networks).

Os autores de~\cite{SocialNetworking} desenvolveram um aplicativo
do Facebook que foi denominado Cityware. A ideia dessa aplicação é
mesclar informações disponíveis no Facebook com traces de
mobilidade capturados através de Bluetooth. Isso proporciona a
diminuição da distância entre as redes sociais online e
físicas. Os usuários possuem a possibilidade explorar uma
combinação de suas redes sociais online com suas redes sociais físicas.

Em~\cite{Hummingbird} os autores apresentam um dispositivo denominado
Hummingbird, que alerta quando existe um usuário perto de outro.
Hummingbird suporta colaboração e foi desenvolvido para melhorar formas
tradicionais de comunicação online, como mensagens instantâneas.

Foram apresentados nessa Seção apenas alguns dos principais trabalhos
encontrados na literatura relacionados ao nosso tema. No decorrer do
trabalho serão revisados outros trabalhos com o intuito de
aprimorar essa Seção.
