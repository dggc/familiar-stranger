\section{Detalhes do \textit{Dataset}}\label{modelagem}

Neste trabalho utilizamos o dataset disponibilizado pelo projeto Reality
Mining. O projeto Reality Mining foi realizado de 2004 a 2005 no
laboratório MIT Media. Esse projeto monitorou 94 usuários usando seus
telefones celulares com um programa específico para gravar diversas
atividades realizadas. Dentre as informações gravadas incluem: ligações
realizadas e recebidas, ID bluetooth de telefones próximos, cell ID da
área que o usuário de encontra em determinado momento, entre outras.

Os usuários monitorados eram estudantes e e funcionários do MIT. Além
desse monitoramento realizado, foi realizado um questionário com os
participantes. Esse questionário também contém várias informações úteis,
como: os amigos de determinada pessoa, hábitos e atividades realizadas
no período do experimento. 

Mais detalhes desse dataset podem ser encontrados na seguinte
referência: \cite{eagle2007isn}.
