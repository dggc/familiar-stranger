\section{Introdução}

Este trabalho trabalho objetivou trabalhar com dois tópicos de pesquisa
de grande repercussão atualmente: redes sociais e network coding
(codificação em rede). Considera-se que uma rede social convencional é
uma rede, por exemplo, nos modelos do Orkut e Facebook.  Uma rede social
móvel, que é o tipo de rede que estamos interessados neste trabalho, se
difere de uma rede social convencional, principalmente, porque o
objetivo nessa última é conectar/aproximar usuários que já estão
fisicamente próximos. Por exemplo, usuários presos em algum
engarrafamento podem interagir com outras pessoas, para tentar tornar
essa situação desagradável em uma situação prazerosa. Outro exemplo
seria uma rede social móvel em um campus universitário. Alunos podem
conhecer outros alunos e, por exemplo, combinar uma carona, supondo que
foi identificado que ambos possuem horários trajetos parecidos.

Uma rede social móvel é uma rede que está sujeita a atrasos e
desconexões frequentes. Esse fenômeno, também chamado de rede DTN ou
oportunista, vem recebendo grande atenção de diversos pesquisadores. Um
dos problemas desse tipo de rede são os altos atrasos que as mensagens
podem sofrer até chegarem no destinatário, uma vez que o roteamento só
ocorre com a proximidade física dos usuários.

Codificação em rede é um técnica particular de processamento de dados na
rede que explora características do meio sem fio, com o objetivo de
aumentar a capacidade ou o throughput da rede. Essa técnica poderia ser
benéfica em redes DTN.

Nesse contexto, a ideia do trabalho é simular uma rede social móvel, com
padrões de movimentos realistas, a fim de analisar o impacto da
utilização de codificação em rede na entrega de mensagens. Para avaliar
essa questão foi utilizado o simulador SINALGO~\cite{sinalgo}. 
